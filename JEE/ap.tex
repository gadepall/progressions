\begin{enumerate}    [label=\thesubsection.\arabic*, ref=\thesubsection.\theenumi]
%
\item  The sum of integers from $1$ to $100$ that are divisible by $2$ or $5$ is \rule{1cm}{0.1pt}.\hfill{(1984)}
		  \item  Let $p$ and $q$ be the roots of the equation                    $${x^2-2x+A=0}$$ and $r$ and $s$ be the roots of the                     equation ${x^2-18x+B=0}$. If ${p<q<r<s}$ are                                      in arithmetic progression,  then find $A$ and $B$. \hfill{(1977)}
\item Let $a_{1},  a_{2},  a_{3}\dots a_{100}$ be an arithmetic progression with $a_{1}= 3$ and $$S_{p} =\sum\limits_{i=1}^{p} a_{i}, 1\leq p\leq 100.$$ For any integer $n$ with $1 \leq n \leq 20, $ let $m= 5n$. If $\frac{S_{m}}{S_{n}}$ does not depend on n,  then $a_{2}$ is \rule{1cm}{0.1pt}.\hfill(2011)
%
   \item A pack contains $n$ cards numbered for 1 to $n$,  two consecutive numbered cards are removed from the pack and then the sum of the on the remaining cards is 1224. If the smaller of the numbers on the removed cards is $k$,  then $k - 20=\rule{1cm}{0.1pt}.$ \hfill(2013)
   \item Suppose all the numbers of an arithmetic progression(AP) are natural numbers. If the ratio of the sum of the first seven terms to the sum of the first eleven terms is 6 : 11 and the seventh term lies between 130 and 140,  then the common difference of the AP is 
    \hfill(2015)
%
   \item The sides of a right angled triangle are in arithmetic progression. If the triangle has area 24,  what is the length of its smallest side? 
%   
   \hfill(2018)
%
   \item Let $X$ be the set consisting of the first 2018 terms of the arithmetic progression 1,  6,  11,  \dots  and $Y$ be the set consisting of the first 2018 terms of the arithmetic progression 9,  16,  23,  \dots Then,  the number of elements in the set $ X \cup Y $ is \rule{1cm}{0.1pt}.\hfill(2018)
   \item Let AP $\brak{a; d}$ denote the set of all the terms of an infinite arithmetic progression with the first term $a$ and the common difference $d>0$. If $$AP\brak{1; 3}AP\brak{2; 5}AP\brak{3; 7}+AP\brak{a; d}$$ then $a + d$ equals \rule{1cm}{0.1pt}.\hfill(2019)
%   
	\item {If $ 1,  \log_9 \brak{3^{1-x} +2},  \log_3 \brak{4\cdot3^x -1}$ are in AP then $x$ equals}
{\hfill{\brak{2002}}}
\begin{multicols}{4}
\begin{enumerate}    
\item  {$\log_3 4$}
 \item {$1-\log_3 4$}
 \item {$1-\log_4 3$}
 \item {$\log_4 3$}
\end{enumerate}
\end{multicols}
%
\item {Let $T_r$ be the $r^{th}$ term of an AP whose first term is a and common difference is $d$. If for some positive integers $m, n,  m\neq n,  T_m = \frac{1}{n}$ and $T_n = \frac{1}{m}$,  then $a-d$ equals} 
{\hfill{\brak{2004}}}
\begin{multicols}{4}
\begin{enumerate}    
\item  {$\frac{1}{m}+\frac{1}{n}$}
\item  {$1$}
\item  {$\frac{1}{mn}$}
\item  {$0$}
\end{enumerate}
\end{multicols}
\item Let $a_1,  a_2,  a_3 \dots$ be terms of an AP. If $\frac{a_1+a_2+\dots a_p}{a_1+a_2+\dots a_q}= \frac{p^2}{q^2},  p \neq q$,  then $\frac{a_6}{a_{21}}$ equals
\hfill \brak{2006}
\begin{multicols}{4}
\begin{enumerate}    
\item  {$\frac{41}{11}$}
\item  {$\frac{7}{2}$}
\item  {$\frac{2}{7}$}
\item  {$\frac{11}{41}$}
\end{enumerate}
\end{multicols}
    \item If $a_1, a_2, \dots, a_n$ are in HP,  then the expression $a_1a_2+a_2a_3+\dots+a_{n-1}a_n$ is equal to

%    
    \hfill(2006)
%
    \begin{multicols}{4}
\begin{enumerate}    
    \item$n(a_1-a_n)$
    \item$(n-1)(a_1-a_n)$
    \item$na_1a_n$
    \item$(n-1)a_1a_n$ 
    \end{enumerate}
\end{multicols}
%    
%17
%18
%
%19
%
%20
%21  
    \item  A person is to count $4500$ currency notes. Let $a_n$ denote the number of notes he counts in the $n^{th}$ minute. If $a_1=a_2=\dots=a_{10}=150$ and $a_{10}, a_{11}, \dots$ are in an AP with common difference $-2$,  then the time taken by him to count all notes is   
%     
    \hfill(2010)
%    
    \begin{multicols}{4}
\begin{enumerate}    
    \item$34$ minutes
    \item$125$ minutes
    \item$135$ minutes
    \item$24$ minutes 
    \end{enumerate}
\end{multicols}
%
%22
    \item A man saves \rupee$200$ in each of the first three months of his service. In each of the subsequent months his saving increases by \rupee$40$ more than the saving of immediately previous month. His total saving from the start of service will be \rupee$11040$ after    
%    
    \hfill(2011)
%    
    \begin{multicols}{4}
\begin{enumerate}    
    \item$19$ months
    \item$20$ months
    \item$21$ months
    \item$18$ months
%
    \end{enumerate}
\end{multicols}
%
%
  \item Let $a, b, c \in R$. If $f(x)=ax^2+bx+c$ is such that $a+b+c=3$ and $$f(x+y)=f(x)+f(y) \forall x, y \epsilon R,$$  then $\sum _{n=1}^{10}  f(n)$  is  equal  to \hfill (2017)
	 \begin{multicols}{4}
\begin{enumerate}    
  \item{255}
  \item{330}
  \item{165}
  \item{190}
  \end{enumerate}
  \end{multicols}
  \item Let $a_{1}, a_{2}, \dots, a_{30}$ be an AP. $S=\sum_{i=1}^{30}a_{i}$ and $T=\sum_{i=2}^{15}a_{(2i-1)}$. If $a_{5}=27$
	  and $S-2T=75$,  then $a_{10}$ is equal to \null \hfill{(2019)}
\begin{multicols}{4}
\begin{enumerate}    
  \item {52} \item{57}
  \item{47}
  \item{42}
  \end{enumerate}
\end{multicols}
  \item Let the sum of the first $n$ terms of a non-constant AP: $a_{1}, a_{2}, a_{3}, \dots $ be $50n + \frac{n(n-7)}{2}A$,  where $A$ is a constant. If $d$ is the common difference of this AP then ordered pair ($d$, $a_{50}$) is equal to \hfill{(2019)} 
	  \begin{multicols}{4}
\begin{enumerate}    
	\item {(50,  $50+46A$)} \item{(50,  $50+45A$)}
  \item{($A$,  $50+45A$)}
  \item{($A$,  $50+46A$)}
  \end{enumerate}
\end{multicols}
%
\item Let $T_r$ be the $r^{th}$ term of an AP,  for $r=1, 2, 3, \dots$ If for some positive integers $m, n$ we have
$T_m=\frac{1}{n}$ and $T_n=\frac{1}{m}$ , then $T_{mn}$ equals \hfill\brak{1998}
%
\begin{multicols}{4}
\begin{enumerate}    
\item $\frac{1}{mn}$
\item $\frac{1}{m} + \frac{1}{m}$
\item $1$
\item $0$
\end{enumerate}
\end{multicols}
\item Let ${a_1, a_2, \dots a_{10}}$ be in AP and ${h_1, h_2,  \dots h_{10}}$ be in HP. If ${a_1}={h_1}=2$ and ${a_{10}}={h_{10}}=3$,  then ${a_4h_7}$ is \hfill(1992 )
    \begin{multicols}{4}
\begin{enumerate}    
        \item 2
        \item 3
        \item 5
        \item 6
        \end{enumerate}
        \end{multicols}
\item The harmonic mean of the roots of the equation
        $$(5+\sqrt{2})x^2-(4+\sqrt{5})x+8+2\sqrt{5}=0$$ is 
        \hfill(1999)
%            
%        
        \begin{multicols}{4}
\begin{enumerate}    
            \item 2
            \item 4
            \item 6
            \item 8
        \end{enumerate}
        \end{multicols}
\item If the sum of the first $2n$ terms of the AP: $2, 5, 8, \dots, $ is equal to the sum of the first $n$ terms of the AP: $ 57, 59, 61, \dots, $ then $n$ equals \hfill(2001)
        \begin{multicols}{4}
\begin{enumerate}    
            \item 10
            \item 12
            \item 11
            \item 13
            \end{enumerate}
            \end{multicols}
\item If the sum of first $n$ terms of an AP is $cn^2$,  then the sum of squares of these $n$ terms is \hfill(2009)
%                
%            
             \begin{multicols}{4}
\begin{enumerate}    
                    \item $\frac{n(4n^2-1)c^2}{6}$
                    \item $\frac{n(4n^2+1)c^2}{3}$
                    \item $\frac{n(4n^2-1)c^2}{3}$
                    \item $\frac{n(4n^2+1)c^2}{6}$
		\end{enumerate}
         \end{multicols}
%  
\item Let ${a_1, a_2, a_3, \dots}$ be in harmonic progression with ${a_1}=5$ and ${a_{20}}=25$. The least positive integer $n$ for which ${a_n<0}$ is \hfill(2012)
                \begin{multicols}{4}
\begin{enumerate}    
                    \item 22
                    \item 23
                    \item 24
                    \item 25
                    \end{enumerate}
                    \end{multicols}
\item Let ${b_i}>1$ for $i=1, 2, \dots, 101$. Suppose ${\log_e}{b_1}, {\log_e}{b_2}, \dots, {\log_e}{b_{101}}$ are in Arithmetic Progression (AP) with the common difference ${\log_e}2$. Suppose ${a_1, a_2, \dots, a_{101}}$ are in AP such that ${a_1=b_1}$ and ${a_{51}=b_{51}}$. If $t={b_1+b_2+\dots+b_{51}}$ and $s={a_1+a_2+\dots+a_{53}}$,  then \hfill ( 2016)
                    \begin{multicols}{2}
\begin{enumerate}    
%                    
                        \item $s>t$ and ${a_{101}>b_{101}}$
                        \item $s>t$ and ${a_{101}<b_{101}}$
                        \item $s<t$ and ${a_{101}>b_{101}}$
                        \item $s<t$ and ${a_{101}<b_{101}}$
                        \end{enumerate}
                        \end{multicols}
%    
\item[]
 Let $ V_{r} $ denote the sum of first $r$ terms of an arithmetic progression  whose first term is $r$ and the common difference is $\brak{2r-1}$. Let $ T_{r}=V_{r+1}-V_{r}-2 $ and $ Q_{r}=T_{r+1}-T_{r}$ for $r=1, 2, \dots$
% 
%
 \item The sum  $  V_{1}+V_{2}+\dots+V_{n} $  is 
% 
	 \hfill \brak{2007 }                            
     \begin{multicols}{2}
\begin{enumerate}    
%         
	     \item $\frac{1}{12}n\brak{n+1}\brak{3n^{2}-n+1}$
	     \item $\frac{1}{12}n\brak{n+1}\brak{3n^{2}+n+2}$
	     \item $\frac{1}{2}n\brak{2n^{2}-n+1}$
	     \item $\frac{1}{3}\brak{2n^{3}-2n+3}$
    \end{enumerate}
\end{multicols} 
%
  \item $T_{r}$ is always 
%                           
	  \hfill \brak{2007 }                    
                  \begin{multicols}{2}      
\begin{enumerate}     
%	
       \item an odd number 
       \item an even number
	\item a prime number 
        \item composite number
%
	  \end{enumerate}
   \end{multicols}
    \item Which one of the following is a correct statement? 
%          
	    \hfill \brak{2007 }                                  
\begin{enumerate}    
	\item $Q_{1}, Q_{2}, Q_{3}, \dots$ are in AP with common difference 5 
	\item $Q_{1}, Q_{2}, Q_{3}, \dots$ are in AP with common difference 6
	\item $Q_{1}, Q_{2}, Q_{3}, \dots$ are in AP with common difference 11
	\item $Q_{1}=Q_{2}=Q_{3}=\dots$
	\end{enumerate}
%       
%
\item The interior angles of a polygon are in arithmetic progression. The smallest angle is $120\degree$ and the common difference is $5\degree$. Find the number of sides of the polygon.
%
\hfill\brak{1980}
%
%
%	    
	     \item If $ \log_{3}{2}, \log_{3}{2^{x}-3} $ and $ \log_{3}{\brak{2^{x}-\frac{7}{2}}} $ are in arithmetic progression, determine the value of $x$.  
%     
	      \hfill \brak{1990 }
%      
%
      \item Let $p$ be the first of $n$ arithmetic means between two numbers and $q$ the first of $n$ harmonic means between the same numbers . Show that $q$ does not lie between $p$ and $\brak{\frac{n+1}{n-1}}^{2}p$ 
%       
	      \hfill \brak{1991 }
%      
%
%       
%
		\item  The real numbers $ x_{1}, x_{2}, x_{3} $ satisfying the equation $ x^{3}-x^{2}+\beta x+\gamma=0 $ are in AP. Find the intervals in which $ \beta $ and $\gamma$ lie.
%       
			\hfill \brak{1996 }
%      
%
%
      \item The fourth power of the common difference of an arithmetic progression with integer entries is added to the product of any four consecutive terms of it. Prove that the resulting sum is the square of an integer.
%      
	      \hfill \brak{2000 }
%    
%
%     
%
\end{enumerate}
