\begin{enumerate}    [label=\thesubsection.\arabic*, ref=\thesubsection.\theenumi]
%
\item  The sum of integers from $1$ to $100$ that are divisible by $2$ or $5$ is \rule{1cm}{0.1pt}.\hfill{(1984)}
%
   \item A pack contains $n$ cards numbered from 1 to $n$,  two consecutive numbered cards are removed from the pack and then the sum of the numbers on the remaining cards is 1224. If the smaller of the numbers on the removed cards is $k$,  then $k - 20=\rule{1cm}{0.1pt}.$ \hfill(2013)
   \item Suppose all the numbers of an (AP) are natural numbers. If the ratio of the sum of the first seven terms to the sum of the first eleven terms is 6 : 11 and the seventh term lies between 130 and 140,  then the common difference of the AP is \rule{1cm}{0.1pt}.
    \hfill(2015)
%
   \item The sides of a right angled triangle are in  AP . If the triangle has area 24,  what is the length of its smallest side? 
%   
   \hfill(2018)
%
   \item Let $X$ be the set consisting of the first 2018 terms of the  AP  1,  6,  11,  \dots  and $Y$ be the set consisting of the first 2018 terms of the  AP  9,  16,  23,  \dots Then,  the number of elements in the set $ X \cup Y $ is \rule{1cm}{0.1pt}.\hfill(2018)
   \item Let AP $\brak{a; d}$ denote the set of all the terms of an infinite  AP  with the first term $a$ and the common difference $d>0$. If $$AP\brak{1; 3}\cap AP\brak{2; 5}\cap AP\brak{3; 7} = AP\brak{a; d}$$ then $a + d$ equals \rule{1cm}{0.1pt}.\hfill(2019)
%   
    \item  A person is to count $4500$ currency notes. Let $a_n$ denote the number of notes he counts in the $n^{th}$ minute. If $a_1=a_2=\dots=a_{10}=150$ and $a_{10}, a_{11}, \dots$ are in an AP with common difference $-2$,  then the time taken by him to count all notes is   
%     
    \hfill(2010)
%    
    \begin{multicols}{4}
\begin{enumerate}    
    \item$34$ minutes
    \item$125$ minutes
    \item$135$ minutes
    \item$24$ minutes 
    \end{enumerate}
\end{multicols}
%
%22
    \item A man saves \rupee$200$ in each of the first three months of his service. In each of the subsequent months his saving increases by \rupee$40$ more than the saving of immediately previous month. His total saving from the start of service will be \rupee$11040$ after    
%    
    \hfill(2011)
%    
    \begin{multicols}{4}
\begin{enumerate}    
    \item$19$ months
    \item$20$ months
    \item$21$ months
    \item$18$ months
%
    \end{enumerate}
\end{multicols}
%
  \item Let $a_{1}, a_{2}, \dots, a_{30}$ be an AP. $S=\sum_{i=1}^{30}a_{i}$ and $T=\sum_{i=2}^{15}a_{(2i-1)}$. If $a_{5}=27$
	  and $S-2T=75$,  then $a_{10}$ is equal to \null \hfill{(2019)}
\begin{multicols}{4}
\begin{enumerate}    
  \item {52} \item{57}
  \item{47}
  \item{42}
  \end{enumerate}
\end{multicols}
  \item Let the sum of the first $n$ terms of a non-constant AP: $a_{1}, a_{2}, a_{3}, \dots $ be $50n + \frac{n(n-7)}{2}A$,  where $A$ is a constant. If $d$ is the common difference of this AP then ordered pair ($d$, $a_{50}$) is equal to \hfill{(2019)} 
	  \begin{multicols}{4}
\begin{enumerate}    
	\item {(50,  $50+46A$)} \item{(50,  $50+45A$)}
  \item{($A$,  $50+45A$)}
  \item{($A$,  $50+46A$)}
  \end{enumerate}
\end{multicols}
%
\item Let ${a_1, a_2, \dots a_{10}}$ be in AP and ${h_1, h_2,  \dots h_{10}}$ be in HP. If ${a_1}={h_1}=2$ and ${a_{10}}={h_{10}}=3$,  then ${a_4h_7}$ is \hfill(1992)
    \begin{multicols}{4}
\begin{enumerate}    
        \item 2
        \item 3
        \item 5
        \item 6
        \end{enumerate}
        \end{multicols}
\item The harmonic mean of the roots of the equation
        $$(5+\sqrt{2})x^2-(4+\sqrt{5})x+8+2\sqrt{5}=0$$ is 
        \hfill(1999)
%            
%        
        \begin{multicols}{4}
\begin{enumerate}    
            \item 2
            \item 4
            \item 6
            \item 8
        \end{enumerate}
        \end{multicols}
\item If the sum of the first $2n$ terms of the AP: $2, 5, 8, \dots, $ is equal to the sum of the first $n$ terms of the AP: $ 57, 59, 61, \dots, $ then $n$ equals \hfill(2001)
        \begin{multicols}{4}
\begin{enumerate}    
            \item 10
            \item 12
            \item 11
            \item 13
            \end{enumerate}
            \end{multicols}
%
\item The interior angles of a polygon are in  AP . The smallest angle is $120\degree$ and the common difference is $5\degree$. Find the number of sides of the polygon.
%
\hfill\brak{1980}
%
%    
\item     Five numbers are in AP, whose sum is 25 and product is 2520. If one of these five numbers is $-\frac{1}{2}$, then the greatest number among them is
\hfill \brak{2020}
%
\begin{multicols}{4}
\begin{enumerate}    
\item     $\frac{21}{2}$
\item     $27$
\item     $16$
\item     $7$
\end{enumerate}
\end{multicols}
%
%     
%
\item     Let $l_1,l_2,\dots,l_{100}$ be consecutive terms of an  AP  with common difference $d_1$, and let $w_1, w_2, \dots , w_{100}$ be consecutive terms of another  AP  with common difference $d_2 $, where $d_1d_2$ = 10. For each $i = 1, 2,\dots,100$, let $R_i$ be a rectangle with length $L_i$, width $W_i$ and area $A_i$. If $A_{51}-A_{50}=1000$, then the value of $A_{100}-A_{90}$ is \rule{1cm}{0.1pt}.
	\hfill \brak{2022}
\item If the sum of the first 40 terms of the series: 3 + 4 + 8 + 9 + 13 + 14 + 18 + 19 + \dots is $102m$, then $m$ is equal to
	\hfill \brak{2024}
		\begin{multicols}{4}
\begin{enumerate}
   \item $20$
   \item $5$
   \item $10$
   \item $25$
\end{enumerate}
                                         \end{multicols} 
					 \item If the $10^{\text{th}}$ term of an  AP  is $\frac{1}{20}$ and its $20^{\text{th}}$ term is $\frac{1}{10}$, then the sum of its first 200 terms is
	\hfill \brak{2020}
		\begin{multicols}{4}
\begin{enumerate}
    \item $50 \frac{1}{4}$
    \item $100 \frac{1}{2}$
    \item $50$
    \item $100$
\end{enumerate}
                                         \end{multicols} 
\item If the sum of the first 40 terms of the series:\begin{align*}3 + 4 + 8 + 9 + 13 + 14 + 18 + 19 + \dots\end{align*}is $(102)m$, then $m$ is equal to
	\hfill \brak{2020}
		\begin{multicols}{4}
\begin{enumerate}
   \item $20$
   \item $5$
   \item $10$
   \item $25$
\end{enumerate}
                                         \end{multicols} 
\end{enumerate}
