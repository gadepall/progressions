\begin{enumerate}[label=\thesubsection.\arabic*,ref=\thesubsection.\theenumi]

\item Let $a_{1},  a_{2},  a_{3}\dots a_{100}$ be an  AP  with $a_{1}= 3$ and $$S_{p} =\sum\limits_{i=1}^{p} a_{i}, 1\leq p\leq 100.$$ For any integer $n$ with $1 \leq n \leq 20, $ let $m= 5n$. If $\frac{S_{m}}{S_{n}}$ does not depend on $n$,  then $a_{2}$ is \rule{1cm}{0.1pt}.\hfill(2011)

		  \item  Let $p$ and $q$ be the roots of the equation                    $${x^2-2x+A=0}$$ and $r$ and $s$ be the roots of the                     equation ${x^2-18x+B=0}$. If ${p<q<r<s}$ are                                      in  AP ,  then find $A$ and $B$. \hfill{(1977)}
	\item {If $ 1,  \log_9 \brak{3^{1-x} +2},  \log_3 \brak{4\cdot3^x -1}$ are in AP then $x$ equals}
{\hfill{\brak{2002}}}
\begin{multicols}{4}
\begin{enumerate}    
\item  {$\log_3 4$}
 \item {$1-\log_3 4$}
 \item {$1-\log_4 3$}
 \item {$\log_4 3$}
\end{enumerate}
\end{multicols}
%
\item {Let $T_r$ be the $r^{th}$ term of an AP whose first term is a and common difference is $d$. If for some positive integers $m, n,  m\neq n,  T_m = \frac{1}{n}$ and $T_n = \frac{1}{m}$,  then $a-d$ equals} 
{\hfill{\brak{2004}}}
\begin{multicols}{4}
\begin{enumerate}    
\item  {$\frac{1}{m}+\frac{1}{n}$}
\item  {$1$}
\item  {$\frac{1}{mn}$}
\item  {$0$}
\end{enumerate}
\end{multicols}
\item Let $a_1,  a_2,  a_3 \dots$ be terms of an AP. If $\frac{a_1+a_2+\dots a_p}{a_1+a_2+\dots a_q}= \frac{p^2}{q^2},  p \neq q$,  then $\frac{a_6}{a_{21}}$ equals
\hfill \brak{2006}
\begin{multicols}{4}
\begin{enumerate}    
\item  {$\frac{41}{11}$}
\item  {$\frac{7}{2}$}
\item  {$\frac{2}{7}$}
\item  {$\frac{11}{41}$}
\end{enumerate}
\end{multicols}
    \item If $a_1, a_2, \dots, a_n$ are in HP,  then the expression $a_1a_2+a_2a_3+\dots+a_{n-1}a_n$ is equal to

%    
    \hfill(2006)
%
    \begin{multicols}{4}
\begin{enumerate}    
    \item$n(a_1-a_n)$
    \item$(n-1)(a_1-a_n)$
    \item$na_1a_n$
    \item$(n-1)a_1a_n$ 
    \end{enumerate}
\end{multicols}
%    
%
  \item Let $a, b, c \in R$. If $f(x)=ax^2+bx+c$ is such that $a+b+c=3$ and $$f(x+y)=f(x)+f(y) \forall x, y \epsilon R,$$  then $\sum _{n=1}^{10}  f(n)$  is  equal  to \hfill (2017)
	 \begin{multicols}{4}
\begin{enumerate}    
  \item{255}
  \item{330}
  \item{165}
  \item{190}
  \end{enumerate}
  \end{multicols}
\item Let $T_r$ be the $r^{th}$ term of an AP,  for $r=1, 2, 3, \dots$ If for some positive integers $m, n$ we have
$T_m=\frac{1}{n}$ and $T_n=\frac{1}{m}$ , then $T_{mn}$ equals \hfill\brak{1998}
%
\begin{multicols}{4}
\begin{enumerate}    
\item $\frac{1}{mn}$
\item $\frac{1}{m} + \frac{1}{m}$
\item $1$
\item $0$
\end{enumerate}
\end{multicols}
\item Let ${a_1, a_2, a_3, \dots}$ be in harmonic progression with ${a_1}=5$ and ${a_{20}}=25$. The least positive integer $n$ for which ${a_n<0}$ is \hfill(2012)
                \begin{multicols}{4}
\begin{enumerate}    
                    \item 22
                    \item 23
                    \item 24
                    \item 25
                    \end{enumerate}
                    \end{multicols}
\item Let ${b_i}>1$ for $i=1, 2, \dots, 101$. Suppose ${\log_e}{b_1}, {\log_e}{b_2}, \dots, {\log_e}{b_{101}}$ are in  AP   with the common difference ${\log_e}2$. Suppose ${a_1, a_2, \dots, a_{101}}$ are in AP such that ${a_1=b_1}$ and ${a_{51}=b_{51}}$. If $t={b_1+b_2+\dots+b_{51}}$ and $s={a_1+a_2+\dots+a_{53}}$,  then \hfill ( 2016)
                    \begin{multicols}{2}
\begin{enumerate}    
%                    
                        \item $s>t$ and ${a_{101}>b_{101}}$
                        \item $s>t$ and ${a_{101}<b_{101}}$
                        \item $s<t$ and ${a_{101}>b_{101}}$
                        \item $s<t$ and ${a_{101}<b_{101}}$
                        \end{enumerate}
                        \end{multicols}
%    
\item[]
 Let $ V_{r} $ denote the sum of first $r$ terms of an  AP   whose first term is $r$ and the common difference is $\brak{2r-1}$. Let $ T_{r}=V_{r+1}-V_{r}-2 $ and $ Q_{r}=T_{r+1}-T_{r}$ for $r=1, 2, \dots$
% 
%
 \item The sum  $  V_{1}+V_{2}+\dots+V_{n} $  is 
% 
	 \hfill \brak{2007 }                            
     \begin{multicols}{2}
\begin{enumerate}    
%         
	     \item $\frac{1}{12}n\brak{n+1}\brak{3n^{2}-n+1}$
	     \item $\frac{1}{12}n\brak{n+1}\brak{3n^{2}+n+2}$
	     \item $\frac{1}{2}n\brak{2n^{2}-n+1}$
	     \item $\frac{1}{3}\brak{2n^{3}-2n+3}$
    \end{enumerate}
\end{multicols} 
%
  \item $T_{r}$ is always 
%                           
	  \hfill \brak{2007 }                    
                  \begin{multicols}{2}      
\begin{enumerate}     
%	
       \item an odd number 
       \item an even number
	\item a prime number 
        \item composite number
%
	  \end{enumerate}
   \end{multicols}
    \item Which one of the following is a correct statement? 
%          
	    \hfill \brak{2007 }                                  
\begin{enumerate}    
	\item $Q_{1}, Q_{2}, Q_{3}, \dots$ are in AP with common difference 5 
	\item $Q_{1}, Q_{2}, Q_{3}, \dots$ are in AP with common difference 6
	\item $Q_{1}, Q_{2}, Q_{3}, \dots$ are in AP with common difference 11
	\item $Q_{1}=Q_{2}=Q_{3}=\dots$
	\end{enumerate}
%       
%
%	    
	     \item If $ \log_{3}{2}, \log_{3}{2^{x}-3} $ and $ \log_{3}{\brak{2^{x}-\frac{7}{2}}} $ are in  AP , determine the value of $x$.  
%     
	      \hfill \brak{1990 }
%      
%
      \item Let $p$ be the first of $n$ arithmetic means between two numbers and $q$ the first of $n$ harmonic means between the same numbers . Show that $q$ does not lie between $p$ and $\brak{\frac{n+1}{n-1}}^{2}p$ 
%       
	      \hfill \brak{1991 }
%      
%
%       
%
		\item  The real numbers $ x_{1}, x_{2}, x_{3} $ satisfying the equation $ x^{3}-x^{2}+\beta x+\gamma=0 $ are in AP. Find the intervals in which $ \beta $ and $\gamma$ lie.
%       
			\hfill \brak{1996 }
%      
%
%
      \item The fourth power of the common difference of an  AP  with integer entries is added to the product of any four consecutive terms of it. Prove that the resulting sum is the square of an integer.
%      
	      \hfill \brak{2000 }
\item     Let $p,q$ amd $r$  be nonzero real numbers that are the $10^{th}, 100^{th},$ and $1000^{th}$ terms of a harmonic progression, respectively. Consider the following system of linear equations

	\hfill \brak{2022}
\begin{align*}
x + y + z &= 1
\\
10x + 100y + 1000z &= 0
\\
qr x + pr y + pq z &= 0
\end{align*}

		\begin{multicols}{2}
			\begin{enumerate}[label=(\Roman*)]		
%\begin{enumerate}		
\item     If $ \frac{q}{r} = 10 $, then the system of linear equations has 

\item      If $ \frac{p}{r} \neq 100 $, then the system of linear equations has 

\item      If $ \frac{p}{q} \neq 10 $, then the system of linear equations has 
\item      If $ \frac{p}{q} = 10 $, then the system of linear equations has 
\end{enumerate}
\columnbreak
					\begin{enumerate}[label=(\Alph*)]		
\item     $ x = 0,  y = \frac{10}{9}, z = -\frac{1}{9} $ as a solution  
\item     $ x = \frac{10}{9},  y = -\frac{1}{9},  z = 0 $ as a solution 
\item     infinitely many solutions                               
\item     no solution 
\end{enumerate}
                                         \end{multicols} 

\begin{enumerate}		
\item     $(I)\to(T);(II)\to(C);(III)\to(D);(IV)\to(T)$
\item     $(I)\to(B);(II)\to(D);(III)\to(D);(IV)\to(C)$   
\item     $(I)\to(B);(II)\to(C);(III)\to(A);(IV)\to(C)$
\item     $(I)\to(T);(II)\to(D);(III)\to(A);(IV)\to(T)$
\end{enumerate}
\end{enumerate}


