\begin{enumerate}[label=\thesubsection.\arabic*,ref=\thesubsection.\theenumi]

\item     Let $l_1,l_2,\dots,l_{100}$ be consecutive terms of an arithmetic progression with common difference $d_1$, and let $w_1, w_2, \dots , w_100$ be consecutive terms of another arithmetic progression with common difference $d_2 $, where $d_1d_2$ = 10. For each $i = 1, 2,\dots,100$, let $R_i$ be a rectangle with length $L_i$, width $W_i$ and area $A_i$. If $A_{51}-A_{50}=1000$, then the value of $A_{100}-A_{90}$ is \rule{1cm}{0.1pt}.
 
	\hfill \brak{2022}
\item     Let $a_1, a_2, a_3,\dots$ be an arithmetic progression with $a_1 = 7$ and common difference 8. Let 
$T_1, T_2, T_3,$\dots be such that $T_1$ = 3 and $T_{n+1} - T_n = a_n$ for $n\ge1$. Then, which of the following is/are TRUE?
	\hfill \brak{2022}
\begin{enumerate}
\item     $T_{20} =1604$  
\item    $\sum_{K=1}^{20}T_k=10510$
\item     $T_{30}=3454$ 
\item     $\sum_{K=1}^{30}T_k=357610$  
\end{enumerate}
\item     Let $p,q$ amd $r$  be nonzero real numbers that are the $10^{th}, 100^{th}, and 1000^{th}$ terms of a harmonic progression, respectively. Consider the following system of linear equations
	\hfill \brak{2022}
\begin{align*}
x + y + z &= 1
\\
10x + 100y + 1000z &= 0
\\
qr x + pr y + pq z &= 0
\end{align*}

		\begin{multicols{2}
\begin{enumerate}[label=\Roman*.]		
\item     If \( \frac{q}{r} = 10 \), then the system of linear equations has 

\item      If \( \frac{p}{r} \neq 100 \), then the system of linear equations has 

\item      If \( \frac{p}{q} \neq 10 \), then the system of linear equations has 
\item      If \( \frac{p}{q} = 10 \), then the system of linear equations has 
\end{enumerate}
\columnbreak
\begin{enumerate}[label=\Alph*.]		
\item     \( x = 0,  y = \frac{10}{9}, z = -\frac{1}{9} \) as a solution  
\item     \( x = \frac{10}{9},  y = -\frac{1}{9},  z = 0 \) as a solution 
\item     infinitely many solutions                               
\item     no solution 
\end{enumerate}
                                         \end{multicols} 

\begin{enumerate}		
\item     $(I)\to(T);(II)\to(C);(III)\to(D);(IV)\to(T)$
\item     $(I)\to(B);(II)\to(D);(III)\to(D);(IV)\to(C)$   
\item     $(I)\to(B);(II)\to(C);(III)\to(A);(IV)\to(C)$
\item     $(I)\to(T);(II)\to(D);(III)\to(A);(IV)\to(T)$
\end{enumerate}

\end{enumerate}


