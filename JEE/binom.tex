\begin{enumerate}[label=\thesubsection.\arabic*,ref=\thesubsection.\theenumi]
%
\item For $0 < \phi < \frac{\pi}{2}$,  if 
\begin{align*}
x=\sum_{n=0}^{\infty} \cos^{2n} \phi ,  \; 
y=\sum_{n=0}^{\infty} \sin^{2n} \phi ,  \; 
z=\sum_{n=0}^{\infty} \cos^{2n} \phi \sin^{2n} \phi
\end{align*}
then \hfill\brak{1993}
\begin{multicols}{2}
\begin{enumerate}    
\item $xyz=xz+y$
\item $xyz=xy+z$
\item $xyz=x+y+z$
\item $xyz+yz+x$
\end{enumerate}
\end{multicols}
\item Let $n$ be a odd integer. If 
\begin{align*}
\sin n\theta= \sum_{r=0}^{n} b_r \sin^{r} \theta,  
\end{align*}
for every value of $\theta$,  then
\hfill\brak{1998}
\begin{multicols}{2}
\begin{enumerate}    
\item $b_0=1,  b_1=3$
\item $b_0=0,  b_1=n$
\item $b_0=-1,  b_1=3$
\item $b_0=0,  b_1=n^2-3n+3$
\end{enumerate}
\end{multicols}
	\item Find the sum of the series 
	    \hfill \brak{1985 }
		\[
\sum_{r=0}^{n} (-1)^r \binom{n}{r} \left( \frac{1}{2^r} + \frac{3^r}{2^{2r}} + \frac{7^r}{2^{3r}} + \frac{15^r}{2^{4r}}  \cdots \text{up to $m$ terms} \right)
\]
%	    
%
\item The larger of $99^{50}+100^{50}$ and $101^{50}$ is \dots \hfill \brak{1982}
\item The sum of the coefficients of the polynomial $\brak{1+x-3x^2}^{2163}$ is \dots \hfill \brak{1982}
\item If $\brak{1+ax}^n=1+8x+24x^2+\dots$ then $a=\dots$ and $n=\dots$
	\hfill \brak{1983}
\item Let $n$ be  a positive integer. If the coefficients of 2nd, 3rd and 4th terms in the expansion of $\brak{1 + x}^n$ are in  AP, then the value of $n$ is \dots	\hfill \brak{1994}
\item The sum of the rational terms in the expansion of $\brak{\sqrt{2}+3^\frac{1}{5}}^{10}$ is \dots
	\hfill \brak{1997}
        \item
            The coefficients of three consecutive terms of $\brak{1+x}^{n+5}$ are in the ratio 5:10:14. Then $n=$       
                     \hfill(2013)
        \item
            Let $m$ be the smallest positive integer such that the coefficient of $x^{2}$ in the expansion of $\brak{1+x}^{2} + \brak{1+x}^{3} + ... + \brak{1+x}^{49} + \brak{1+x}^{50} + \brak{1+mx}^{50}$ is $\brak{3n+1} \comb{51}{3}$ for some positive integer $n$. Then the value of $n$ is 
                     \hfill(2016)
        \item
            Let $$X= \brak{\comb{10}{1}^{2}}+2\brak{\comb{10}{2}^{2}} + 3\brak{\comb{10}{3}^{2}} + \dots + 10\brak{\comb{10}{10}^{2}},$$ where $\comb{10}{r} , r \in \cbrak{1,2,\dots ,10}$ denote binomial coefficients. Then, the value of $\frac{X}{1430}$ is \rule{1cm}{0.01pt} 
                    \hfill(2018)
        \item
        Suppose 
           $$\mydet{ \sum\limits_{k=0}^{n}k & \sum\limits_{k=0}^{n} k^{2} \comb{n}{k} \\ \sum\limits_{k=0}^{n}\comb{n}{k} k & \sum\limits_{k=0}^{n}\comb{n}{k} 3^{k}}=0$$ 
holds for some positive integer $n$. The $\sum\limits_{k=0}^{n} \frac{\comb{n}{k}}{k+1}$ equals \rule{10mm}{0.1pt}. 
                    \hfill(2019)
	\item The coefficients of $x^p$ and $x^q$ in the expansion of $\brak{1+x}^{p+q}$ are
	\hfill{(2002)}
	\begin{enumerate}
		\begin{multicols}{2}
		\item equal
		\item equal with opposite signs
		\item reciprocals of each other
		\item none of these
		\end{multicols}
	\end{enumerate}
\item If the sum of coefficients in the expansion of $\brak{a+b}^n$ is 4096, then the greatest coefficient in the expansion is
	\hfill{\brak{2002}}
	\begin{enumerate}
			\begin{multicols}{4}
		\item $1594$
		\item $792$
		\item $924$
		\item $2924$
			\end{multicols}
	\end{enumerate}
\item $r$ and $n$ are positive integers, $r>1, n>2$ and coefficient of $\brak{r+2}^{th}$ term and $\brak{3r}^{th}$ term in the expansion of $\brak{1+x}^{2n}$ are equal, then $n$ equals

	\hfill{\brak{2002}}
	\begin{enumerate}
			\begin{multicols}{4}
		\item $3r$
		\item $3r+1$
		\item $2r$
		\item $2r+1$
			\end{multicols}
	\end{enumerate}
\item The number of integral terms in the expansion of $\brak{\sqrt{3}+\sqrt[8]{5}}^{256}$ is
	\hfill{\brak{2003}}
	\begin{enumerate}
			\begin{multicols}{4}
		\item $35$
		\item $32$
		\item $33$
		\item $34$
			\end{multicols}{4}
	\end{enumerate}
\item The positive integer just greater than 
$(1+0.0001)^{10000}$ is 
		\hfill{\brak{2002}}
	\begin{enumerate}
			\begin{multicols}{4}
		\item $4$
		\item $5$
		\item $2$
		\item $3$
			\end{multicols}
	\end{enumerate}
\item If $x$ is positive, the first negative term in the expansion of $\brak{1+x}^\frac{27}{5}$ is
	\hfill{\brak{2003}}
	\begin{enumerate}
			\begin{multicols}{4}
				\item $6^{th}$ term
		\item $7^{th}$ term
		\item $5^{th}$ term
		\item $8^{th}$ term
			\end{multicols}
	\end{enumerate}
	\item If $C_r$ stands for $\comb{n}{r}$, then the sum of the series $\frac{2\brak{\frac{n}{2}!}\brak{\frac{n}{2}!}}{n!}\sbrak{C_0^2-2C_1^2+3C_2^2-\dots+\brak{-1}^n\brak{n+1}C_n^2}$, where $n$ is an even positive integer is equal to\hfill(1992)
\begin{enumerate}
\begin{multicols}{2}
\item 0
\item $\brak{-1}^{\frac{n}{2}}\brak{n+1}$
\item $\brak{-1}^{\frac{n}{2}}\brak{n+2}$
\item $\brak{-1}^n n$
\item none of these
\end{multicols}
\end{enumerate}
\item If $a_n = \sum_{r=0}^{n}\frac{1}{\comb{n}{r}}$, then $\sum_{r=0}^{n}\frac{r}{\comb{n}{r}}$ equals \hfill(1998)
\begin{enumerate}
\begin{multicols}{2}
\item $\brak{n-1}a_n$
\item $na_n$
\item $\frac{1}{2}na_n$
\item None of the above
\end{multicols}
\end{enumerate}
\item   Given positive integers $r>1,n>2$ and that the coefficient of the $\brak{3r}^{th}$ terms in the binomial expansion of $\brak{1+x}^{2n}$ are equal. Then\hfill(1983)
\begin{enumerate}\begin{multicols}{4}
\item $n=2r$
\item $n=2r+1$
\item $n=3r$
\item none of these
\end{multicols}
\end{enumerate}
% q2
\item The coefficient of $x^4$ in $\brak{\frac{x}{2}-\frac{3}{x^2}}^{10}$ is \hfill(1983)
	\begin{enumerate}[itemsep=1ex]
\begin{multicols}{4}
\item $\frac{405}{256}$
\item $\frac{504}{259}$
\item $\frac{450}{263}$
\item none of these
\end{multicols}
\end{enumerate}
% q3
\item The expression $\brak{x+\brak{x^3-1}^{\frac{1}{2}}}^5$ + $\brak{x-\brak{x^3-1}^{\frac{1}{2}}}^5$ is a polynomial of degree\hfill(1992)
\begin{enumerate}
\begin{multicols}{4}
    \item 5
    \item 6
    \item 7
    \item 8
    \end{multicols}
\end{enumerate}
%q4
\item If in the expansion of $\brak{1+x}^m\brak{1-x})^n$, the coefficients of $x$ and $x^2$ are $3$ and $-6$ respectively, then $m$ is
\hfill(1999)
\begin{enumerate}
\begin{multicols}{4}
    \item 6
    \item 9
    \item 12
    \item 24
    \end{multicols}
\end{enumerate}
%q5
\item For $2\leq r\leq n$, $\comb{n}{r}$ + $2$ $\comb{n}{r-1}$ + $\comb{n}{r-2}$ =\hfill(2000)
\begin{enumerate}\begin{multicols}{2} 
    \item $\comb{n+1}{r-1}$ \item $2$ $\comb{n+1}{r+1}$
    \item $2$ $\comb{n+2}{r}$  \item $\comb{n+2}{r}$
    \end{multicols}
\end{enumerate}
%q6
\item In the binomial expansion of $\brak{a-b}^n$, $n\geq 5$, the sum of  the $5^{th}$ and $6^{th}$ terms is zero. Then $a/b$  equals
\hfill(2001)
\begin{enumerate}
\begin{multicols}{4}
    \item $\frac{n-5}{6}$ 
    \item $\frac{n-4}{5}$
    \item $\frac{5}{n-4}$ 
    \item $\frac{6}{n-5}$
 \end{multicols}   
\end{enumerate}
%q7
\item The sum $\sum_{i=0}^{9} \comb{10}{i}\comb{20}{m-i}$, 
 (where $\comb{p}{q}=0$ if {
$p<q$)} is maximum when $m$ is \hfil(2002)
\begin{enumerate}
\begin{multicols}{4}
    \item5 
    \item10
    \item15
    \item 20
    \end{multicols}
\end{enumerate}
%q8
\item Coefficient of $t^{24}$ in $\brak{{1+t^2}}^{12}\brak{1+t^{12}}\brak{1+t^{24}}$ is \hfill(2003)
\begin{enumerate}
\begin{multicols}{4}
\item $\comb{12}{6}$+3
\item $\comb{12}{6}$+1 
\item $\comb{12}{6}$
\item $\comb{12}{6}$+2\end{multicols}
\end{enumerate}
%q9
\item If 
\begin{align}
\comb{n-1}{r} &= \brak{k^2-3} \comb{n}{r+1} \notag
\end{align}
then ($k \in $ )\hfill(2004)
\begin{enumerate}
\begin{multicols}{4}
    \item $(-8,-2]$
    \item $[2,\infty)$
    \item$\sbrak{-\sqrt{3},\sqrt{3}}$
    \item$(\sqrt{3},2]$
    \end{multicols}
\end{enumerate}

\item The value of
	$\comb{30}{0}$$\comb{30}{10}$-$\comb{30}{1}$$\comb{30}{11}$+$\comb{30}{2}$$\comb{30}{12}$\dots$\comb{30}{20}$$\comb{30}{30}$ is \hfill(2005)
\begin{enumerate}
\begin{multicols}{4}
\item$\comb{30}{10}$ 
\item$\comb{30}{15}$
\item$\comb{60}{30}$ 
\item$\comb{31}{10}$
\end{multicols}
\end{enumerate}
\item For $r=0,1\dots,10$, let $A_r,B_r$ and $C_r$ denote, respectively, the coefficients of $x^r$ in the expansions of $\brak{1+x}^{10}$,$\brak{1+x}^{20}$and $\brak{1+x}^{30}$. Then $\sum_{r=1}^{10}A_r\brak{B_{10}B_r-C_{10}A_r}$ is equal to\hfill(2010)
\begin{enumerate}
 \begin{multicols}{4}
 \item $B_{10}-C_{10}$ 
 \item $A_{10}\brak{B_{10}^2C_{10}A_{10}}$ 
 \item $0$
 \item $C_{10}-B_{10}$
 \end{multicols}
\end{enumerate}

\item  Coefficient of $x^{11}$ in the expansion of $ \brak{1+x^2}^4 \brak{1+x^3}^7 \brak{1+x^4}^{12}$ is \hfill(2014)

\begin{enumerate}
\begin{multicols}{2}
 \item 1051
 \item 1106
 \item 1113
 \item 1120
 \end{multicols}
\end{enumerate}

 \item Given that  \hfill{(1979)} 
			  \begin{align*}
C_1 + 2C_2x + 3C_3x^2 + \dots + 2nC_{2n}x^{2n-1}   =    2n(1+x)^{2n-1} 
			  \end{align*}
		where $C_r =\frac{\brak{2n}!}{r!\brak{2n-r}!}, r=0,1,2,\dots,2n$.     Prove that  \begin{align*} 
		     C^2_1-2C_2^2+3C_3^2-\dots-2nC_{2n}^2  = (-1)^nnC_n. 
		     \end{align*}          
 \item Prove that $ 7^{2n} + \brak{2^{3n-2}}  \brak{3^{n-1}} $ is divisible by 25 for any natural number $ n$. \hfill{(1982)}  
\item If $ \brak{1+x}^n= C_0 + C_1x +C_2x^2+......+C_nx^n $ then show that the sum of products of $ C_i $'s taken
		   two at a time, represented by $ \sum\limits_{0 \leq i<j \leq n}^{}\sum C_i C_j $ is equal to $ 2^{2n-1}$-$\frac{\brak{2n}!} {2\brak{n!}^2} $ \hfill{(1983)}
 \item If $ p $ be a natural number then prove that $ p^{n+1}+\brak{p+1}^{2n-1} $ is divisible by $ p^2+p+1 $ for every positive integer $n$. \hfill{(1984)} 
 \item Given  $ s_n = 1 + q + q^2 +....+q^n;$
		    $ S_n = 1 + \frac{q+1}{2}+\brak{\frac{q+1}{2}}^2+.....+\brak{\frac{q+1}{2}}^n,q\neq1 $ . Prove that
		     \begin{align*} 
			    \comb{n+1}{1}+\comb{n+1}{2}s_1+\comb{n+1}{3}s_2+.....+\comb{n+1}{n}s_n=2^nS_n
		    \end{align*}    \hfill{(1984)}

  \item Let $ R =\brak{5\sqrt{5}+11}^{2n} $ and $ f = R -\sbrak{R} $, where \sbrak{} denotes the greatest integer function.Prove  that $ Rf =4^{2n+4 } $  \hfill {(1988)}
\item Prove that \hfill{(1989)}
		    \begin{align*} 
		    C_0-2^2C_1+3^2C_2-............+\brak{-1}^n\brak{n+1}^2C_n =0
		    \end{align*},$n>2$ , where $C_r =\comb{n}{r}$
		    

 \item Prove that   $ \frac{n^7}{7}+\frac{n^5}{5}+\frac{2n^3}{3}-\frac{n}{105}$ is an integer for every positive integer $ n$. \hfill{(1990)}

 \item If $ \sum\limits_{r=0}^{2n} a_r \brak{x-2}^r=\sum\limits_{r=0}^{2n}b_r \brak{x-3}^r $ and $ a_k =1 $ for all $k \geq  n$ then show that $ b_n = \comb{2n+1}{n+1} $ \hfill{(1992)}

\item If $a_n = \sqrt{7+\sqrt{7+\sqrt{7+...}}}$ having $n$ radical signs, then by methods of mathematical induction, which is true?
	\hfill{\brak{2002}}
	\begin{enumerate}
			\begin{multicols}{2}
		\item $a_n > 7$  $\forall$ $n \ge 1$
		\item $a_n < 7$  $\forall$ $n \ge 1$
		\item $a_n < 4$  $\forall$ $n \ge 1$
		\item $a_n < 3$  $\forall$ $n \ge 1$
			\end{multicols}
	\end{enumerate}
\item If $a_n = \sqrt{7+\sqrt{7+\sqrt{7+...}}}$ having $n$ radical signs, then by methods of mathematical induction, which is true?
	\hfill{\brak{2002}}
	\begin{enumerate}
			\begin{multicols}{2}
		\item $a_n > 7$  $\forall$ $n \ge 1$
		\item $a_n < 7$  $\forall$ $n \ge 1$
		\item $a_n < 4$  $\forall$ $n \ge 1$
		\item $a_n < 3$  $\forall$ $n \ge 1$
			\end{multicols}
	\end{enumerate}
 \item Use mathematical Induction to prove : If $n$ is any odd positive integer , then  $ n\brak{n^2-1} $ is divisible by 24.
		   \hfill{(1983)}
 \item Use method of mathematical Induction  $ 2.7^n +3.5^n-5 $ is divisible by 24 for all $ n>0 $ \hfill{(1985)}
 \item Prove by mathematical induction that -$\frac{\brak{2n}!}{2^{2n}\brak{n!}^2}\leq \frac{1}{\brak{3n+1}^{\frac{1}{2}}} $ for all postive Integers $n$.\hfill {(1987)}
 \item Using mathematical induction,prove that
		    \begin{align*} \comb{m}{0}\comb{n}{k} +\comb{m}{1}\comb{n}{k-1}+\dots..+\comb{m}{k}\comb{n}{0} =\comb{m+k}{k} \end{align*}  \hfill{(1989)}
 \item Using induction or otherwise , prove that for any non-negative integers $m,n,r$ and $ k$,
		    \begin{align*} 
			    \sum_{r=0}^{k}(n-m)\frac{(r+m)!}{m!}= \frac{(r+k+1)!}{k!}\sbrak{\frac{n}{r+1}-\frac{k}{r+2}}
		    \end{align*}
		     \hfill{(1991)} 
  \item Let $ p \leq 3 $ be an integer and $ \alpha , \beta $ be  the roots of $ x^2-\brak{p+1}x+1=0 $ using mathematical induction show that $ \alpha^n  + \beta^n $ \hfill{(1992)}
		    \begin{enumerate}[label=(\roman*)]

			    \item is an integer and   
		    \item is not divisible by $ p $ \end{enumerate}
				    
\end{enumerate}

