\begin{enumerate}[label=\thesubsection.\arabic*,ref=\thesubsection.\theenumi]
    \item The sum of series $\frac{1}{2!}-\frac{1}{3!}+\frac{1}{4!}-\dots$ upto infinity is 
    \hfill(2007)
%
    \begin{enumerate}
    \item$e^{-\frac{1}{2}}$
    \item$e^{+\frac{1}{2}}$
    \item$e^{-2}$
    \item$e^{-1}$
    \end{enumerate}
%
	 \item A man $X$ has $7$ friends,  $4$ of them are ladies and $3$ are men. His wife $Y$ also has $7$ friends,  $3$ of them are ladies and $4$ are men. Assume $X$ and $Y$ have no common friends.Then the total number of ways in which $X$ and $Y$ together can throw a party inviting $3$ ladies and $3$ men,  so that $3$ friends of each of $X$ and $Y$ are in this party,  is \hfill{(2017)}
\begin{multicols}{4}
\begin{enumerate}    
\item $484$ 
\item $485$
\item $468$
\item $469$
\end{enumerate}
\end{multicols}
%
	\item From $6$ different novels and $3$ different dictionaries, $4$ novels and $1$ dictionary are to be selected and arranged in a row on a shelf so that  the dictionary is always in the middle. The number of such arrangements is:\hfill{(2018)}
 \begin{multicols}{4}
\begin{enumerate}    
     \item less than $500$ 
     \item at least $500$ but less than $750$
     \item at least $750$ but less than $1000$
     \item at least $1000$
     \end{enumerate}
\end{multicols}
%
	\item Consider a class of $5$ girls and $7$ boys. The number of different teams consisting of $2$ girls and $3$ boys that can be formed from this class, if there are two specific boys $A$ and $B$, who refuse to be members of the same team, is:\hfill{[2019-9 Jan(M)]}
\begin{multicols}{4}
\begin{enumerate}    
    \item $500$  
    \item $200$
    \item $300$
    \item $350$
    \end{enumerate}
\end{multicols}
%
	\item A committee of $11$ members is to be formed from $8$ males and $5$ females. If m is the number of ways the committee is formed with at least $6$ males and n is the number of ways the committee is formed with at least $3$ females,  is:
		\hfill{[2019-9April(M)]}
\begin{multicols}{4}
\begin{enumerate}    
      \item $m+n=68$ 
      \item $m=n=78$
      \item $n=m-8$
      \item $m=n=68$
  \end{enumerate}
\end{multicols}  
\item Three circles of radii $a,  b,  c$ ($a<b<c$) touch each other externally. If they have x-axis as a common tangent,  then:\hfill{(2019)}\begin{multicols}{4}
\begin{enumerate}     \itemsep.5em	
%  
  \item {$\frac{1}{\sqrt{a}}=\frac{1}{\sqrt{b}}+\frac{1}{\sqrt{c}}$} 
  \item {$\frac{1}{\sqrt{b}}=\frac{1}{\sqrt{c}}+\frac{1}{\sqrt{a}}$} 
  \item$a, b$ and $c$ are in AP
  \item{${\sqrt{a}}={\sqrt{b}}+{\sqrt{c}}$}
  \end{enumerate}
\end{multicols} 
\item For $0 < \phi < \pi /2$,  if 
\begin{align*}
x=\sum_{n=0}^{\infty} \cos^{2n} \phi ,  \; 
y=\sum_{n=0}^{\infty} \sin^{2n} \phi ,  \; 
z=\sum_{n=0}^{\infty} \cos^{2n} \phi \sin^{2n} \phi
\end{align*}
then: \hfill\brak{1993}
\begin{multicols}{4}
\begin{multicols}{4}
\begin{enumerate}    
\item $xyz=xz+y$
\item $xyz=xy+z$
\item $xyz=x+y+z$
\item $xyz+yz+x$
\end{enumerate}
\end{multicols}
\end{multicols}
\item Let $n$ be a odd integer. If 
\begin{align*}
\sin n\theta= \sum_{r=0}^{n} b_r \sin^{r} \theta,  
\end{align*}
for every value of $\theta$,  then
\hfill\brak{1998}
\begin{multicols}{2}
\begin{multicols}{4}
\begin{enumerate}    
\item $b_0=1,  b_1=3$
\item $b_0=0,  b_1=n$
\item $b_0=-1,  b_1=3$
\item $b_0=0,  b_1=n^2-3n+3$
\end{enumerate}
\end{multicols}
\end{multicols}
%
\item A straight line through the vertex $\vec{P}$ of a triangle $\Delta$ intersects the side QR at the point $\vec{S}$ and the circumcircle of the triangle $\Delta$ at the point $\vec{T}$. If $\vec{S}$ is not the centre of the circumcircle, then  \hfill\brak{2008}
\begin{multicols}{2}
\begin{multicols}{4}
\begin{enumerate}    
\item $\frac{1}{PS}+\frac{1}{ST}<\frac{2}{\sqrt{QS \cdot SR}}$
\item $\frac{1}{PS}+\frac{1}{ST}>\frac{2}{\sqrt{QS \cdot SR}}$
\item $\frac{1}{PS}+\frac{1}{ST}<\frac{4}{QR}$
\item $\frac{1}{PS}+\frac{1}{ST}>\frac{4}{QR}$
\end{enumerate}
\end{multicols}
\end{multicols}
\item The number ${\log_2}7$ is \hfill(1990)
    \begin{multicols}{2}
    \begin{multicols}{4}
\begin{enumerate}    
        \item an integer
        \item a rational number
        \item an irrational number
        \item A prime number
    \end{enumerate}
\end{multicols}
    \end{multicols}
    \item If $a>0, b>0, c>0$,  prove that
    \\
    $\brak{a+b+c}\brak{\frac{1}{a}+\frac{1}{b}+\frac{1}{c}}\geq9$
    \hfill\brak{1984}
%
    \item If N is a natural number such that
\\ 
$n= p_{1}^{a_1}.p_{2}^{a_2}.p_{3}^{a_3}......p_{k}^{a_k} $ and $ p_{1}, p_{2}, ...., p_{k} $ are distinct primes,  then show that $ ln n \geq k ln2 $                              
%     
		\hfill \brak{1985}              
%
	\item Find the sum of the series :\[
\sum_{r=0}^{n} (-1)^r \binom{n}{r} \left( \frac{1}{2^r} + \frac{3^r}{2^{2r}} + \frac{7^r}{2^{3r}} + \frac{15^r}{2^{4r}}  \cdots \text{up to m terms} \right)
\]
%
%
%
%	    
%	    
	    \hfill \brak{1985 }
     \item Solve for $x$ the following equation:     
%     
	     \hfill \brak{1987}          \\              
		     $   \log_{\brak{2x+3}}{\brak{6x^{2}+23x+21}}=4-\log_{\brak{3x+7}}{\brak{4x^{2}+12x+9}} $
%
\item The number of integers greater than $6, 000$ that can be formed,  using digits $3$, $5$, $6$, $7$ and $8$,  without repetition,  is\hfill{(2015)}
    \begin{multicols}{4}
\begin{enumerate}    
    \item $120$ 
    \item $72$
    \item $216$
    \item $192$ 
    \end{enumerate}
\end{multicols} 
%    
	    \item If all words (with or without) having five letters, formed using the letters of the word SMALL and arranged as in a dictionary; then the position of the word SMALL is 

		    \hfill{(2015)}
\begin{multicols}{4}
\begin{enumerate}    
    \item $ 52^{nd} $
    \item $ 58^{th} $
    \item $ 46^{th} $
    \item $ 59^{th} $
    \end{enumerate}
\end{multicols} 
\item For a positive integer $n$,  let
$a_n=1+\frac{1}{2}+\frac{1}{3}+\frac{1}{4}+\dots\frac{1}{(2^n)-1}$. Then \hfill\brak{1999}
\begin{multicols}{4}
\begin{enumerate}    
\item $a100)\leq 100$
\item $a(100) > 100$
\item $a(200)\leq 100$
\item $a(200) > 100$
\end{enumerate}
\end{multicols}
%
%
\item Let 
\begin{align*}
S_n=\sum_{k=1}^{n}\frac{n}{n^2+kn+k^2} \text{ and }   T_n=\sum_{k=0}^{n-1}\frac{n}{n^2+kn+k^2}
\end{align*}
for $n=1, 2, 3, \dots$ Then, \hfill\brak{2008}
\begin{multicols}{4}
\begin{enumerate}    
\item $S_n<\frac{\pi}{3\sqrt{3}}$
\item $S_n>\frac{\pi}{3\sqrt{3}}$
\item $T_n<\frac{\pi}{3\sqrt{3}}$
\item $T_n>\frac{\pi}{3\sqrt{3}}$
\end{enumerate}
\end{multicols}
%
\end{enumerate}
