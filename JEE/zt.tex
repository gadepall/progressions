\begin{enumerate}[label=\thesubsection.\arabic*,ref=\thesubsection.\theenumi]
	     \item The sum of the first $n$ terms of the series ${1^2+2.2^2+3^2+2.4^2+5^2+2.6^2+\dots}$ is
   ${n (n+1)^2 /2}$, when $n$ is even. When $n$ is odd, the sum 
   is\dots\hfill{(1988)}

		  \item For any odd integer $n \ge 1$, ${n^3-(n-1)^3+(-1)^{n-1} 1^3=\dots}$\hfill{(1996)}
\item Let $S_{k}, k = 1,2, \dots , 100$, denote the sum of the infinite geometric series whose first term is  $\frac{k - 1}{k!}$ and the common ratio is $\frac{1}{k}$. Then the value of $$\frac{100^{2}}{100!} + \sum\limits_{k=1}^{100} \abs{\brak{k^{2} - 3k +1}S_{k}} $$ is \rule{1cm}{0.1pt}. \hfill(2010)
\item let $a_{1},a_{2},a_{3},\dots , a_{11}$ be real numbers satisfying $a_{1}= 15, 27 - 2a_{2} > 0$ and $a_{k}=2a_{k-1} - a_{k-2}$ for $k=3,4 \dots 11$. If $$\frac{a_{1}^{2} + a_{2}^{2} + \dots +a_{11}^{2}}{11} = 90$$ then the value of $$\frac{a_{1} + a_{2} +\dots +a_{11}}{11}$$ is equal to \rule{1cm}{0.1pt}. \hfill(2010)
\item {The value of $2^{\frac{1}{4}}\cdot 4^{\frac{1}{8}}\cdot 8^{\frac{1}{16}} \ldots \infty$ is}
{\hfill{\brak{2002}}} 
\begin{enumerate}
\begin{multicols}{4}
\item  {$1$}
\item  {$2$}
\item  {$\frac{3}{2}$}
\item  {$4$}
\end{multicols}
\end{enumerate}
\item {Sum of infinite number of terms of a GP is $20$ and sum of their squares is $100$. The common ratio of GP is}
{\hfill{\brak{2002}}} 
\begin{enumerate}
\begin{multicols}{4}

\item  {$5$}
\item  {$\frac{3}{5}$}
\item  {$\frac{8}{5}$}
\item  {$\frac{1}{5}$}
\end{multicols}
\end{enumerate} 

\item {$1^{3}-2^{3}+3^{3}-4^{3}+...
+9^{3}=$}
{\hfill{\brak{2002}}} 
\begin{enumerate}
\begin{multicols}{2}
\item  {$425$}
\item  {$-425$}
\item  {$475$}
\item  {$-475$}
\end{multicols}
\end{enumerate}
\item {The sum of the first $n$ terms of the series $1^2+2\cdot2^2+3^2+2\cdot4^2+5^2+2\cdot6^2+\cdots$ is $\frac{n\brak{n+1}^2}{2}$ when $n$ is even. When $n$ is odd the sum is}
{\hfill{\brak{2004}}}
\begin{enumerate}
\begin{multicols}{4}
\item  {$\sbrak{\frac{n\brak{n+1}}{2}}^2$}
\item  {$\frac{n^2\brak{n+1}}{2}$}
\item  {$\frac{n\brak{n+1}^2}{4}$}
\item  {$\frac{3n\brak{n+1}}{2}$}
\end{multicols}
\end{enumerate}
\item {If $x$ = $\sum\limits_{n=0}^{\infty}a^n$, $y$ = $\sum\limits_{n=0}^{\infty}b^n$, $z$ = $\sum\limits_{n=0}^{\infty}c^n$ where $a,b,c$ are in AP and $\abs{a}<1,\abs{b}<1,\abs{c}<1$ then $x,y,z$ are in}
{\hfill{\brak{2005}}} 
\begin{multicols}{2}
\begin{enumerate}
\item  {GP}
\item  {AP}
\item  {Arithmetic - Geometric Progression}
\item  {HP}
\end{enumerate}
\end{multicols}
    \item The sum of first $9$ terms of the series:
	    $\frac{1^3}{1}+\frac{1^3+2^3}{1+3}+\frac{1^3+2^3+3^3}{1+3+5}+\dots$
    \hfill(2015)
    \begin{multicols}{4}
\begin{enumerate}    
    \item $142$
    \item $192$
    \item $71$
    \item $96$
    \end{enumerate}
\end{multicols}
    \item The sum to infinite term of the series $1+\frac{2}{3}+\frac{6}{3^2}+\frac{10}{3^3}+\frac{14}{3^4}+\dots$ is
    \hfill(2009)
    \begin{multicols}{4}
\begin{enumerate}    
    \item $3$
    \item $4$
    \item $6$
    \item $2$
    \end{enumerate}
\end{multicols}
%
    \item 
    Statement-1 : The sum of the series $$1+(1+2+4)+(4+6+9)+(9+12+16)+\dots+(361+380+400)$$ is $8000$.    
  \\ 
    Statement-2 : $$\sum\limits_{k=1}^n\brak{k^3-\brak{k-1}^3}=n^3, $$ for any natural number $n$.
    \hfill(2012)
\begin{enumerate}    
    \item Statement-1 is false,  Statement-2 is true.
    \item Statement-1 is true;  Statement-2 is true;  Statement-2 is a correct explanation for Statement-1
    \item Statement-1 is true;  Statement-2 is true;  Statement-2 is not a correct explanation for Statement-1
    \item Statement-1 is true;  Statement-2 is false.
    \end{enumerate}
%        
%24
    \item The sum of fîrst $20$ terms of the sequence $0.7, 0.77, 0.777, \dots$ is 
    \hfill(2013)
%    
    \begin{multicols}{4}
\begin{enumerate}    
    \item$\frac{7}{81}\brak{179-10^{-20}}$
    \item$\frac{7}{9}\brak{99-10^{-20}}$
    \item$\frac{7}{81}\brak{179+10^{-20}}$
    \item$\frac{7}{9}\brak{99+10^{-20}}$
    \end{enumerate}
\end{multicols}
%
%25
    \item If $(10)^9+2(11)^1(10)^8+3(11)^2(10)^7+\dots+10(11)^9=k(10)^9$,  then $k$ is equal to
%    
    \hfill(2014)
    \begin{multicols}{4}
\begin{enumerate}    
    \item$100$
    \item$110$
    \item$\frac{121}{10}$
    \item$\frac{441}{100}$ 
    \end{enumerate}
\end{multicols}
%
    \item If the sum of the first ten terms of the series $\brak{1\frac{3}{5}}^2+\brak{2\frac{2}{5}}^2+\brak{3\frac{1}{5}}^2+4^2+\brak{4\frac{4}{5}}^2+\dots$,  is $\frac{16}{5}m$,  then $m$ is equal to 
%    
    \hfill(2016)
    \begin{multicols}{4}
\begin{enumerate}    
    \item$100$
    \item$99$
    \item$102$
    \item$101$
    \end{enumerate}
\end{multicols}
%
% 
	\item If,  for positive integer $n$,  the quadratic equation,  $$x(x+1)+(x+1)(x+2)+\dots+(x+\overline{n-1})(x+n)=10n$$  has two consecutive integral solutions,  then $n$ is equal to     \hfill{(2017)}  
\begin{multicols}{4}
\begin{enumerate}    
    \item {11}
    \item{12}
    \item {9} 
    \item{10}\end{enumerate}
    \end{multicols}
%
  \item Let $a_{1}, a_{2}, a_{3}, \dots, a_{49}$ be an AP such that $\sum_{k=0}^{12} a_{4k+1}=416$ and $a_{9}+a_{43}=66$. If $a_{1}^2+ a_{2}^{2}+\dots+a_{17}^{2}=140m$,  then $m$ is equal to \hfill{(2018)}\begin{multicols}{4}
\begin{enumerate}    
  \item {68}\item {34}\item{33}  \item{66}
  \end{enumerate}
\end{multicols}
  \item Let $A$ be the sum of the frst 20 terms and $B$ be the sum of the first 40 terms of the series $1^{2} +2\cdot2^{2}+3^{2}+2\cdot4^{2}+5^{2}+2\cdot6^{2}+\dots$. If $B-2A=100\lambda$,  then $\lambda$ can be 
	 
	  \hfill (2018)
	  \begin{multicols}{4}
\begin{enumerate}    
  \item {248} \item{464}
  \item{496}
  \item{232}
  \end{enumerate}
\end{multicols}
%
\item Let \begin{align*} S_n=\sum_{k=1}^{4n}\brak{-1}^\frac{k\brak{k+1}}{2}k^2.\end{align*}  Then $S_n$ can take value  \hfill\brak{2013}
\begin{multicols}{4}
\begin{enumerate}    
\item $1056$
\item $1088$
\item $1120$
\item $1332$
\end{enumerate}
\end{multicols}
%
\item Let $\alpha$ and $\beta$ be the roots of $x^2-x-1=0$,  with $\alpha>\beta$. For all positive integers $n$,  define
\begin{align*}
a_n=\frac{\alpha_n-\beta_n}{\alpha-\beta}, n\geq2
b_1=1  \text{ and }  b_n=a_{n-1}+a_{n+1}, n\geq1
\end{align*}
Then which of the following options is/are correct?
\hfill\brak{ 2019}
\begin{multicols}{2}
\begin{enumerate}    
\item $\sum_{n=1}^{\infty}\frac{a_n}{10^n}=\frac{10}{89}$
\item $B_n=a^n+b^n \;  \forall \;  n\geq1$
\item $a_1+a_2+a_3+\dots+a_n=a_{n+2}-1 \forall n\geq1$
\item $\sum_{n=1}^{\infty}\frac{b_n}{10^n}=\frac{8}{89}$
\end{enumerate}
\end{multicols}
%
	\item The rational number,  which equals the number 2.357 with recurring decimal is 
		\hfill{(1983)}
    \begin{multicols}{4}
\begin{enumerate}     \itemsep 1ex
        \item $\frac{2355}{1001}$ 
        \item $\frac{2379}{997}$ 
        \item $\frac{2355}{999}$ 
        \item none of these 
    \end{enumerate}
\end{multicols}
    \item Sum of the first $ n$ terms of the series 
$$ \frac{1}{2}+ \frac{3}{4}+ \frac{7}{8}+ \frac{15}{16}+ \dots $$ is equal to\hfill (1998)
\begin{multicols}{4}
\begin{enumerate}    
    \item $2^n-n-1$  
    \item $1-2^{-n}$
    \item $n+2^{-n}-1$
    \item $2^n+1$
    \end{enumerate}
    \end{multicols}
%
	\item If $ a_n=\frac{3}{4}-\brak{\frac{3}{4}}^{2}+\brak{\frac{3}{4}}^{3}+\dots\brak{-1}^{n-1}\brak{\frac{3}{4}}^{n} $ and $ b_{n}=1-a_{n},  $ then find the least natural number $n_{0}$ such that $ b_{n} > a_{n} \forall n \geq n_{0}. $ 
%		             
		\hfill \brak{2006 }
    \item  In a geometric progression consisting of positive terms,  each term equals the sum of the next two terms. Then the common ratio of its progression equals   
    \hfill(2007)
%    
    \begin{multicols}{4}
\begin{enumerate}    
    \item$\sqrt{5}$
    \item$\frac{1}{2}\brak{\sqrt{5}-1}$
    \item$\frac{1}{2}\brak{1-\sqrt{5}}$
    \item$\frac{1}{2}\sqrt{5}$ 
    \end{enumerate}
\end{multicols}
      \item  If $ S_1, S_2, S_3, \dots, S_n $ are the sums of infinite geometric series whose first terms are $1, 2, 3, \dots, n$ and whose common ratios are $ \frac{1}{2}, \frac{1}{3}, \frac{1}{4}, \dots, \frac{1}{n+1} $ respectively,  then find the values of $ S_{1}^{2}+S_{2}^{2}+S_{3}^{2}+\dots+S_{2n-1}^{2} $
%      
	      \hfill \brak{1991 }
\item     Let $a_1, a_2, a_3,\dots$ be an arithmetic progression with $a_1 = 7$ and common difference 8. Let 
$T_1, T_2, T_3,$\dots be such that $T_1$ = 3 and $T_{n+1} - T_n = a_n$ for $n\ge1$. Then, which of the following is/are TRUE?
	\hfill \brak{2022}
		\begin{multicols}{2}
\begin{enumerate}
\item     $T_{20} =1604$  
\item    $\sum_{K=1}^{20}T_k=10510$
\item     $T_{30}=3454$ 
\item     $\sum_{K=1}^{30}T_k=357610$  
\end{enumerate}
                                         \end{multicols} 
\item If the sum of first $n$ terms of an AP is $cn^2$,  then the sum of squares of these $n$ terms is \hfill(2009)
%                
%            
             \begin{multicols}{4}
\begin{enumerate}    
                    \item $\frac{n(4n^2-1)c^2}{6}$
                    \item $\frac{n(4n^2+1)c^2}{3}$
                    \item $\frac{n(4n^2-1)c^2}{3}$
                    \item $\frac{n(4n^2+1)c^2}{6}$
		\end{enumerate}
         \end{multicols}
%  
\end{enumerate}
