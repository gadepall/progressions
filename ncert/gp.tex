\begin{enumerate}[label=\thesubsection.\arabic*.,ref=\thesubsection.\theenumi]
\item Find the $20^{th}$ and $n^{th}$ terms of the GP: $\frac{5}{2}, \frac{5}{4}, \frac{5}{8},\dots, $
	\\
	\solution
\begin{align}
	x_0 &= \frac{5}{2}, r = \frac{1}{2}
	\\
	\implies x_{19} &=\frac{5}{2}\brak{\frac{1}{2}}^{19}=\frac{5}{2^{20}}
	\\
	x_{n-1} &= \frac{5}{2^n}
\end{align}
using
	\eqref{eq:gp-nthterm}.
\item The $4^{th}$ term of a GP  is square of its second term, and the first term is -3. Determine its $7^{th}$ term.
	\\
	\solution  From the given information, 
\begin{align}
	x_3 &= x_1^2, x_0 = -3
	\\
	\implies x_0r^3&= x_0^2r^2
	\\
	\text{or, } r &= x_0
	\\
	\therefore x_6 &= x_0^7= \brak{-3}^7
\end{align}
\item Which term of the following sequences
\begin{enumerate}
	\item 2, 2 $\sqrt{2}$, 4,\dots,  is 128 ?
	\item $\sqrt{3}, 3, 3\sqrt{3},\dots,$  is 729?
	\item $\frac{1}{3}, \frac{1}{9}, \frac{1}{27}, \dots,$  is $\frac{1}{19683}$?
\end{enumerate}
	\solution
\begin{enumerate}
	\item 
\begin{align}
	x_0&= 2, r = \sqrt{2}
	\\
	x_n &= x_0r^n = 128
	\\
		\implies 2\brak{\sqrt{2}}^n &= 128
		\\
		\text{or, } n &= 2\brak{\frac{\log 128}{\log 2}-1}=12
\end{align}
\end{enumerate}
\item For what values of $x$, the numbers $-\frac{2}{7}, x, -\frac{7}{2}$ are in GP?
\\
	Find the sum to indicated number of terms in each of the geometric progressions.
\item 0.15, 0.015, 0.0015,\dots,  20 terms.
\item $\sqrt{7}, \sqrt{21}, \sqrt[3]{7}, \dots,  n$ terms.
\item $1, -a, a^2, -a^2, a^3,\dots,  n$ terms $\brak{a \neq -1}$.
\item $x^3, x^5, x^7,\dots,  n$ terms $\brak{x \neq \pm 1}$.
\item Evaluate $$\sum _{k=1}^{11} \brak{2 + 3^k}.$$
	\\
	\solution  The summation can be expressed as
\begin{align}
\sum _{k=1}^{11} 2 +\sum _{k=1}^{11} 3^k
	=2\times 11 +\sum _{k=0}^{10} 3^{k+1}
	\\
	=22+3\sum _{k=0}^{10} 3^{k}=22+3\brak{\frac{3^{11}-1}{2}}
\end{align}
\item The sum of first three terms of a GP  is $\frac{39}{10}$ and their product is 1. Find the 
common ratio and the terms.
	\\
	\solution
\begin{align}
	x_0\brak{1+r+r^2}=\frac{39}{10}
	\\
	x_0^3r^3=1 \implies x_0r = 1
	\\
	\therefore
	{10}\brak{1+r+r^2}={39}r
\end{align}
yielding the quadratic
\begin{align}
	{10}r^2-29r + 10&=0
	\\
	\implies \brak{5r - 2}\brak{2r-5}&=0
	\\
	\text{or, }
	r = \frac{5}{2},&\frac{2}{5}
\end{align}
\item How many terms of GP  3, $3^2, 3^3$, \dots,  are needed to give the sum 120?
\item The sum of first three terms of a GP  is 16 and the sum of the next three terms is 128. Determine the first term, the common ratio and the sum to $n$ terms of the GP 
\item Given a GP  with $a = 729$ and $7^{th}$ term 64, determine $y_6$.
\item The number of bacteria in a certain culture doubles every hour. If there were 30 bacteria present in the culture originally, how many bacteria will be present at the end of $2^{nd}$ hour, 
$4^{th}$ hour and $n^{th}$ hour ?
\item What will Rs 500 amount to in 10 years after its deposit in a bank which pays annual interest rate of 10\% compounded annually?
\item If AM  and GM  of roots of a quadratic equation are 8 and 5, respectively, then obtain the quadratic equation.
	\\
	\solution If the roots are $a, b$,
\begin{align}
	\frac{a+b}{2} = 8, \implies a+b = 16,
	\\
	\sqrt{ab} = 5 \implies ab = 25
	\\
	\therefore 
	x^2-16x + 25=0
\end{align}
is the desired quadratic equation.
\item The sum of some terms of GP  is 315 whose first term and the common ratio are 5 and 2, respectively. Find the last term and the number of terms.
\item  The first term of a GP  is 1. The sum of the third term and fifth term is 90. Find the common ratio of the GP.
\item The sum of three numbers in GP  is 56. If we subtract 1, 7, 21 from these numbers in that order, we obtain an arithmetic progression. Find the numbers.
\item A person writes a letter to four of his friends. He asks each one of them to copy
the letter and mail to four different persons with instruction that they move the
chain similarly. Assuming that the chain is not broken and that it costs 50 paise to
mail one letter. Find the amount spent on the postage when $8^{th}$ set of letter is
mailed. 
\item A manufacturer reckons that the value of a machine,  which costs him Rs. 15625,  will depreciate each year by 20\%. Find the estimated value at the end of 5 years. 
\item 150 workers were engaged to finish a job in a certain number of days. 4 workers dropped out on second day,  4 more workers dropped out on third day and so on.It took 8 more days to finish the work. Find the number of days in which the work was completed.
	\item Find the $10^{th}$ and $n^{th}$ and  terms of the GP: $5, 25, 125, \dots$
	\item Which term of the GP: $2, 8, 32, \dots $ upto $n$ terms is 131072.
	\item In a GP the $3^{rd}$ term is 24 and the $6^{th}$ term is 192.  Find the $10^{th}$ term.
	\item Find the sum of the first $n$ terms and the sum of the first 5 terms of the series 
		$$ 1 + \frac{2}{3}+\frac{4}{9}+\dots$$
	\item How many terms of the GP: 
		$$ 3 + \frac{3}{2}+\frac{3}{4}+\dots$$
		are needed to give the sum $\frac{3069}{512}$.
	\item The sum of the first 3 terms of a GP is 
$\frac{13}{12}$ and their product is -1.  Find the common ratio and the terms.
\item A person has 2 parents, 4 grandparents, 8 great grand parents and so on.  Find the number of his ancestors during the ten generations preceeding his own.
\item Insert 3 numbers between 1 and 256 so that the resulting sequence is a GP.
\item If the AM and GM of two positive numbers $a$ and $b$ are 10 and 8 respectively, find the numbers.
\item Find the $12^{th}$ term of a GP  whose $8^{th}$ term is 192 and the common ratio is 2.
\item Find a GP  for which sum of the first two terms is -4 and the fifth term is 4 times the third term.
\item Find four numbers forming a geometric progression in which the third term is greater than the first term by 9, and the second term is greater than the $4^{th}$ by 18.
\end{enumerate}
